% Options for packages loaded elsewhere
\PassOptionsToPackage{unicode}{hyperref}
\PassOptionsToPackage{hyphens}{url}
%
\documentclass[
]{article}
\usepackage{lmodern}
\usepackage{amssymb,amsmath}
\usepackage{ifxetex,ifluatex}
\ifnum 0\ifxetex 1\fi\ifluatex 1\fi=0 % if pdftex
  \usepackage[T1]{fontenc}
  \usepackage[utf8]{inputenc}
  \usepackage{textcomp} % provide euro and other symbols
\else % if luatex or xetex
  \usepackage{unicode-math}
  \defaultfontfeatures{Scale=MatchLowercase}
  \defaultfontfeatures[\rmfamily]{Ligatures=TeX,Scale=1}
\fi
% Use upquote if available, for straight quotes in verbatim environments
\IfFileExists{upquote.sty}{\usepackage{upquote}}{}
\IfFileExists{microtype.sty}{% use microtype if available
  \usepackage[]{microtype}
  \UseMicrotypeSet[protrusion]{basicmath} % disable protrusion for tt fonts
}{}
\makeatletter
\@ifundefined{KOMAClassName}{% if non-KOMA class
  \IfFileExists{parskip.sty}{%
    \usepackage{parskip}
  }{% else
    \setlength{\parindent}{0pt}
    \setlength{\parskip}{6pt plus 2pt minus 1pt}}
}{% if KOMA class
  \KOMAoptions{parskip=half}}
\makeatother
\usepackage{xcolor}
\IfFileExists{xurl.sty}{\usepackage{xurl}}{} % add URL line breaks if available
\IfFileExists{bookmark.sty}{\usepackage{bookmark}}{\usepackage{hyperref}}
\hypersetup{
  pdftitle={Do companies Headquartered in privacy-friendly jurisdictions have better privacy policies?},
  pdfauthor={Pseudo McNamesely},
  hidelinks,
  pdfcreator={LaTeX via pandoc}}
\urlstyle{same} % disable monospaced font for URLs
\usepackage[margin=1in]{geometry}
\usepackage{graphicx}
\makeatletter
\def\maxwidth{\ifdim\Gin@nat@width>\linewidth\linewidth\else\Gin@nat@width\fi}
\def\maxheight{\ifdim\Gin@nat@height>\textheight\textheight\else\Gin@nat@height\fi}
\makeatother
% Scale images if necessary, so that they will not overflow the page
% margins by default, and it is still possible to overwrite the defaults
% using explicit options in \includegraphics[width, height, ...]{}
\setkeys{Gin}{width=\maxwidth,height=\maxheight,keepaspectratio}
% Set default figure placement to htbp
\makeatletter
\def\fps@figure{htbp}
\makeatother
\setlength{\emergencystretch}{3em} % prevent overfull lines
\providecommand{\tightlist}{%
  \setlength{\itemsep}{0pt}\setlength{\parskip}{0pt}}
\setcounter{secnumdepth}{-\maxdimen} % remove section numbering
\ifluatex
  \usepackage{selnolig}  % disable illegal ligatures
\fi

\title{Do companies Headquartered in privacy-friendly jurisdictions have
better privacy policies?}
\author{Pseudo McNamesely}
\date{09/04/2021}

\begin{document}
\maketitle
\begin{abstract}
The past five years have marked a significant shift in the work of
privacy policy researchers. Enabled by the Usable Privacy Project,
myriad datasets in the form of annotated privacy policy corpora have
been developed and further research is only accellerating. This project
briefly tracks the efforts made by The Usable Privacy Project before
turning to an experiment of it's own, which aims to measure whether
emerging privacy legislation such as the General Data Protection
Regulation (GDPR) and the California Consumer Privacy Act (CCPA) have an
impact on the quality of privacy policies. Using a multiple linear
regression, this project aims to speak on potential causality between
the legal jurisdiction where a company is headquartered and the quality
of their privacy polict. At this time, you are reading a sentence that
holds the place of where I will briefly talk about the result. Feel free
to read this sentence in the voice of Morgan Freeman, let his warm
baritone wash over you and make you remember a time when you didnt have
so much work to do.
\end{abstract}

\hypertarget{introduction}{%
\section{Introduction}\label{introduction}}

Over the past five years, research in the communities of digital privacy
and data protection have taken unprecedented strides. While this is
strongly reflected in recent actions taken by governments to revamp
their digital privacy legislation, the vastness of platforms with their
own privacy policies provides a wellspring of data for researchers to
dive into. The most significant research program in this field is
undoubtedly the Universal Privacy Policy Project {[}sadeh{]}, which
marks an international and multidisciplinary effort to collect the data
found within privacy policies as corpora, and then provide these corpora
to researchers for further data analysis. Work on this project began in
2013, and to date seven data sets and 62 publications have been produced
that cover relevant areas such as scaling the collection of privacy
policy corpora {[}MAPS{]}, the reliability of machine learning models in
classifying specific ``data actions'' within corpora {[}kumar{]}, and
empirically measuring the privacy risks of a representative sample of
privacy policies {[}Bhatia{]}.

The following project and experimental design draws heavily on the
methods and techniques developed by the Usable Privacy Project in
addition to utilizing the Usable Privacy datasets for further
statistical analysis. Namely, the data used in this experiment is the
OPP-115 corpus, which is collection of 115 annotated privacy policies
that were pre-selected through sector-based sub-sampling by the
principal researchers of this project {[}Wilson{]}. Given the relative
complexity of this dataset from gathering through to analysis and
modeling, this project also provides a brief survey of literature
relevant to the Usable Privacy Project as to better acquaint the reader
with methods considered s tate-of-the-art in digital privacy research.

While our internet, and the platforms that collect data from it, operate
on a global scale, the means by which the internet is regulated still
mostly falls into the hands of national actors working within specified
jurisdictions. While researchers in the field of political science and
international relations have an extremely firm understanding of how this
can lead to policy inconsistency in other spaces that require collective
action, relatively little is written on this relationship as it speaks
to privacy and data protection. Although this project is not necessarily
concerned with addressing issues of collective action, it is certainly
concerned with investigating the relationship between regulators who
have taken action and regulators who have not. More specifically, this
project is guided by the following research question:

\textbf{Does the legal jurisdiction of an organization's headquarters
impact the quality of their privacy policies?}

This research question uses somewhat indirect variables to accomplish
it's objective, but can be generally thought of as a way to measure how
effective digital privacy legislation is at creating high quality
privacy policies, which are considered the \emph{de jure} standard for
notifying Internet users of applicable privacy practices" {[}MAPS{]}.
Although there are some limitations to this approach, the experiment
will be conducted utilizing metrics from {[}Wilson{]} et al., that gauge
the frequency at which particular data actions appear in a privacy
policy. Using these frequency percentages, an aggregated continuous
variable \texttt{privacy\_score} will serve as the device to measure
privacy policies. Collecting the legal jurisdictions where each company
was done through basic research of each company and represents our
independent variable \texttt{hq\_jurisdiction}, which is added to the
dataset alongside \texttt{privacy\_score}. For basic non-parametric
estimation, t-tests were employed as the the dependent variable in
question is continuous while the independent variable is categorical.

\[
t = \frac{m-\mu }{s\sqrt{n}} 
\]

To investigate causality, the method used will be a multiple linear
regression, which allows us to better observe a continuous variable
against a categorical variable and provides more metrics for us to base
our inferences off of. A statistical expression of this model can be
understood as follows:

\[
y = \alpha + \beta_{jurisdiction1}x_{jurisdiction1} + \beta_{jurisdiction2}x_{jurisdiction2}+... \beta_{jurisdictionx}x_{jurisdictionx} 
\] It is important to keep this variable \texttt{jurisdiction} rather
than `country' as legal jurisdictions responsible for regulating digital
privacy vary by country. This is highly visible in the European Union's
supranational General Data Protection Regulation (GDPR) and California's
sub-national California Consumer Protection Act (CCPA), both of which
now act as significant components to digital privacy online but neither
operate at a national level. The ambiguity of jurisdiction is another
limitation of this experiment. This limitation, along with others and a
more detailed breakdown of modeling are included in the following
sections {[}@citeR{]}.

\hypertarget{primer-on-the-usable-privacy-project}{%
\section{Primer on The Usable Privacy
Project}\label{primer-on-the-usable-privacy-project}}

\hypertarget{data}{%
\section{Data}\label{data}}

\hypertarget{exploratory-analysis}{%
\section{Exploratory analysis}\label{exploratory-analysis}}

\hypertarget{modeling}{%
\section{Modeling}\label{modeling}}

\hypertarget{results-and-discussion}{%
\section{Results and Discussion}\label{results-and-discussion}}

\hypertarget{closing-remarks}{%
\section{Closing Remarks}\label{closing-remarks}}

\hypertarget{references}{%
\section{References}\label{references}}

\hypertarget{appendices}{%
\section{Appendices}\label{appendices}}

\end{document}
